\chapter{Glossario}
\fancyhead[RO, LE]{\bfseries Glossario}

Spiegazione di alcune sigle utilizzate nei vari capitoli.
\newline
Fonti:
\begin{verbatim}
https://en.wikipedia.org/wiki/
https://docs.conda.io/en/latest/
https://docs.microsoft.com/
\end{verbatim}

\textbf{\textit{EPR}}: The Einstein–Podolsky–Rosen paradox (EPR paradox) is a thought experiment proposed by physicists Albert Einstein, Boris Podolsky and Nathan Rosen (EPR), with which they argued that the description of physical reality provided by quantum mechanics was incomplete. In a 1935 paper titled "Can Quantum-Mechanical Description of Physical Reality be Considered Complete?", they argued for the existence of "elements of reality" that were not part of quantum theory, and speculated that it should be possible to construct a theory containing them. Resolutions of the paradox have important implications for the interpretation of quantum mechanics.

\textbf{\textit{API}}: An application programming interface is a computing interface which defines interactions between multiple software intermediaries. It defines the kinds of calls or requests that can be made, how to make them, the data formats that should be used, the conventions to follow, etc. It can also provide extension mechanisms so that users can extend existing functionality in various ways and to varying degrees.

\textbf{\textit{Conda}}: Conda is an open source package management system and environment management system that runs on Windows, macOS and Linux. Conda quickly installs, runs and updates packages and their dependencies. Conda easily creates, saves, loads and switches between environments on your local computer. It was created for Python programs, but it can package and distribute software for any language.

\textbf{\textit{WSL2}}: The Windows Subsystem for Linux lets developers run a GNU/Linux environment -- including most command-line tools, utilities, and applications -- directly on Windows, unmodified, without the overhead of a traditional virtual machine or dualboot setup.
\newline\newline
Spiegazione di alcuni ottimizzatori utilizzati nella fase sperimentale.
\newline
Fonte: documentazione di Qiskit, reperibile al seguente indirizzo
\begin{verbatim}
https://qiskit.org/documentation.
\end{verbatim}

\textbf{\textit{COBYLA}}: Constrained Optimization By Linear Approximation optimizer. COBYLA is a numerical optimization method for constrained problems where the derivative of the objective function is not known.

\textbf{\textit{SPSA}}: Simultaneous Perturbation Stochastic Approximation (SPSA) optimizer. SPSA is an algorithmic method for optimizing systems with multiple unknown parameters. As an optimization method, it is appropriately suited to large-scale population models, adaptive modeling, and simulation optimization. SPSA is a descent method capable of finding global minima, sharing this property with other methods as simulated annealing. Its main feature is the gradient approximation, which requires only two measurements of the objective function, regardless of the dimension of the optimization problem.

\textbf{\textit{SLSQP}}: Sequential Least SQuares Programming optimizer. SLSQP minimizes a function of several variables with any combination of bounds, equality and inequality constraints. The method wraps the SLSQP Optimization subroutine originally implemented by Dieter Kraft. SLSQP is ideal for mathematical problems for which the objective function and the constraints are twice continuously differentiable. Note that the wrapper handles infinite values in bounds by converting them into large floating values.
