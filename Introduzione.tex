\chapter*{Introduzione}
\fancyhead[RO, LE]{\bfseries Introduzione}

\pagenumbering{arabic}
Questa tesi nasce con l'intento di analizzare l'applicazione pratica di un computer quantistico, in collaborazione con il gruppo \textit{Computational Dynamics and Kinetics} del Dipartimento di Chimica, Biologia e Biotecnologie dell'Università di Perugia, coordinato dalla Dott.ssa Noelia Faginas Lago.  In particolare si vuole misurare l'energia di legame della molecola in funzione dell'interdistanza atomica tra gli atomi costituenti.

La molecola che analizzeremo principalmente sarà l'Idrogeno ($H_2$) in quanto, grazie alla sua semplice struttura, non richiederà tempi di esecuzione troppo lunghi.

Nel raggiungere questo obbiettivo si è voluto far conoscere come con il passare del tempo, a seguito di continue ricerche e studi, si è arrivati dai computer classici attualmente utilizzati, allo studio e realizzazione di un computer quantistico. Negli ultimi quattro anni, dopo l'annuncio da parte di IBM di una linea di prodotti di computer quantistici, c'è stata una forte crescita di prodotti apparsi nel mercato e un enorme e crescente interesse su questa tematica.

È stato necessario quindi descrivere le caratteristiche di un computer quantistico, a partire dal concetto base di qubit e di porta logica quantistica, fino ad arrivare alle classi di complessità e all'algoritmo VQE, che utilizzeremo nella fase sperimentale.

Infine si descrive il framework utilizzato che ci ha permesso di simulare il comportamento di un computer quantistico per i nostri test, restituendo quindi grafici e risultati.