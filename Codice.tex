\chapter{Codice}\label{codice:vqe}
\fancyhead[RO, LE]{\bfseries Codice}
Nel paragrafo seguente viene riportato il codice Python necessario per svolgere una simulazione sulla molecola d'Idrogeno.
Verrà utilizzato il circuito TwoLocal in coppia con l'ottimizzatore SLSQP; il backend Aer ci consentirà di specificare la direttiva \textit{statevector\_gpu} per l'utilizzo della scheda video.

\section{H2 - TwoLocal - SLSQP - Aer - GPU}
\begin{footnotesize}
\begin{minted}[linenos=tru]{python}
from time import time

import matplotlib.pyplot as plt
import numpy as np

from qiskit import Aer
from qiskit.aqua import aqua_globals, QuantumInstance
from qiskit.aqua.algorithms import NumPyMinimumEigensolver, VQE
from qiskit.aqua.components.optimizers import SLSQP
from qiskit.circuit.library import TwoLocal
from qiskit.chemistry.drivers import PySCFDriver, UnitsType
from qiskit.chemistry.core import Hamiltonian, QubitMappingType

start = 0.2
step = 0.05
end = 3.5
d = np.arange(start, end + step, step)
molecule = "H .0 .0 .0; H .0 .0 {0}"
energies = np.zeros(d.size)
hf_energies = np.zeros_like(energies)
distances = np.zeros_like(energies)

# NumPyMinimumEigensolver
for i, v in enumerate(d):
    driver = PySCFDriver(atom=molecule.format(v),
    unit=UnitsType.ANGSTROM,
    basis='sto3g')
    
    qmolecule = driver.run()
    operator = Hamiltonian(qubit_mapping=QubitMappingType.PARITY,
                           two_qubit_reduction=False)
    qubit_op, aux_ops = operator.run(qmolecule)
    result = NumPyMinimumEigensolver(qubit_op).run()
    result = operator.process_algorithm_result(result)
    energies[i] += result.energy
    hf_energies[i] += result.hartree_fock_energy
    distances[i] += v
    
print('Distances: ', distances)
print('Energies:', energies)
print('Hartree-Fock energies:', hf_energies)

plt.figure(figsize=(15,10))
plt.plot(distances, hf_energies,
label="Hartree-Fock")
plt.plot(distances, energies,
label="MinimumEigensolver")

plt.xlabel("Interatomic distance (Angstrom)")
plt.ylabel("Energy (Hartree)")
plt.legend()
plt.grid()

# VQE
aqua_globals.random_seed = 50

energies = np.zeros(d.size)
hf_energies = np.zeros_like(energies)
distances = np.zeros_like(energies)

qi = QuantumInstance(Aer.get_backend('statevector_simulator'),
                    backend_options={"method":"statevector_gpu"},
                    seed_simulator=aqua_globals.random_seed,
                    seed_transpiler=aqua_globals.random_seed)

start_time = time()
for i, v in enumerate(d):
    driver = PySCFDriver(atom=molecule.format(v),
                        unit=UnitsType.ANGSTROM,
                        basis='sto3g')
    
    qmolecule = driver.run()
    operator = Hamiltonian(qubit_mapping=QubitMappingType.PARITY,
                           two_qubit_reduction=False)
    qubit_op, aux_ops = operator.run(qmolecule)
    optimizer = SLSQP(maxiter=250)
    var_form = TwoLocal(qubit_op.num_qubits, ['ry', 'rz'],
                        'cz', reps=5)
    algo = VQE(qubit_op, var_form, optimizer)
    result = algo.run(qi)
    result = operator.process_algorithm_result(result)
    energies[i] += result.energy
    hf_energies[i] += result.hartree_fock_energy
    distances[i] += v

execution_time = (time() - start_time)

print('Distances: ', distances)
print('Energies:', energies)
print('Hartree-Fock energies:', hf_energies)
print('Execution time (minutes):', (execution_time/60))

var_form.draw(output="mpl")

plt.figure(figsize=(15,10))
plt.plot(distances, hf_energies,
label="Hartree-Fock")
plt.plot(distances, energies,
label="VQE")

plt.xlabel("Interatomic distance (Angstrom)")
plt.ylabel("Energy (Hartree)")
plt.legend()
plt.grid()

# Python 3.8.5
# Qiskit version
import qiskit
print(qiskit.__qiskit_version__)

'''
{'qiskit-terra': '0.15.2',
 'qiskit-aer-gpu': '0.6.1',
 'qiskit-ignis': '0.4.0',
 'qiskit-ibmq-provider': '0.10.0',
 'qiskit-aqua': '0.7.5',
 'qiskit': '0.22.0'}
'''
\end{minted}
\end{footnotesize}

